
The rapid advancement of Machine Learning is unlocking unprecedented opportunities to extract critical insights from vast amounts of data. By pursuing a PhD, I aim to deepen my expertise in Data Science and research innovative applications of Machine Learning to uncover ``hidden'' truths and solve real-world problems, such as enhancing quality of life and promoting energy sustainability. I plan to bring the knowledge and skills I will acquire during my doctoral studies in the industry, to contribute to the development of cutting-edge technologies that have a positive impact on society.

My motivation for this pursuit stems from my master's thesis, ``Study and Development of a Digital Twin for the Simulation of User-Created Automations in a Green Smart Home''. This collaborative research project between the University of Brescia and the CNR of Pisa aimed to create a digital twin of a home to track the energy consumption of appliances through trigger-action routines. I extracted information about the appliances' operation modes from raw energy consumption data using state-of-the-art data mining techniques. I then developed a web application that tracks energy consumption, simulates various automation scenarios, and offers energy-saving suggestions. This work resulted in a conference paper for the International Conference on Advanced Visual Interfaces 2024 (AVI 2024) and a journal article published in Future Internet, providing me with valuable publication experience.

I recently completed my master's degree in Computer Science and Engineering at the University of Brescia, where I built a solid foundation in Machine Learning, Data Mining, and Deep Learning techniques and libraries such as TensorFlow and HuggingFace Transformers. My passion for these fields led me to participate in several Kaggle competitions, applying my knowledge to practical problems. I also completed courses on Operational Research and Optimization Algorithms, culminating in a group project that involved tuning the Kernel Search heuristic algorithm to produce high-quality solutions for the Multiple Knapsack Problem.

During my studies, I had the privilege of residing at the Collegio Universitario Luigi Lucchini, a college of merit offering a unique training and cultural program. Balancing my university coursework with the Collegio's demanding schedule was challenging yet rewarding. The seminars, laboratories, and social events allowed me to engage with a community of brilliant minds, teaching me effective communication and teamwork skills. Additionally, traveling abroad for meetings with students from other European colleges broadened my perspectives. These experiences have equipped me to engage meaningfully with peers and effectively manage the demands of research, coursework, and other responsibilities during my doctoral studies.

One area of research I am particularly interested in exploring during my PhD is the intersection of Natural Language Processing (NLP) and cybersecurity. The increasing sophistication of cyber threats necessitates innovative solutions to detect and prevent attacks. NLP techniques can analyze vast amounts of text data to identify patterns and predict potential threats.

Another intriguing topic is the application of machine learning techniques to mobility data to optimize transportation systems, reduce traffic congestion, improve urban planning, and promote sustainable mobility. Data gathered from black boxes inside cars can be used to analyze travel patterns, predict traffic congestion, and optimize public transportation routes.

\iffalse
I am especially interested in exploring how pre-trained LLM models, such as BERT and GPT, can be adapted to cybersecurity applications. For example, these models could be fine-tuned and possibly enhanced through a Retrieval-Augmented Generation (RAG) architecture to detect and respond to cyber threats in real-time, identify security bugs in codebases, or analyze logs for anomalous behavior.
\fi
